%
% Define new commands/macros and/or shortcuts:
%
%     => basic new command:
%          ---\newcommand{what-to-call-command}{what-it-does}
%
%     => arguments sent with [num], where num is how many arguments
%          each argument is addressed using #1 and #2:
%          ---\newcommand{\iso}[2]{$^{#1}$#2}
%             so: \iso{13}{C} = $^{13}$C or \iso{23}{Mg} = $^{23}$Mg
%
%     => optional arguments sent with two brackets [num][val]
%          where num is still number of args, but first argument is given 
%          default value of val if it is not present, note use of [] brackets:
%          ---\newcommand{\carbon}[1][12]{$^{#1}$C}
%             so: \carbon = $^{12}$C but \carbon[13] = $^{13}$C
%          ---\newcommand{\iso}[2][C]{$^{#2}$#1}
%             so: \iso{12} = $^{12}$C but \iso[Mg]{24} = $^{24}$Mg
%
%     => \ensuremath{} allows commands to operate in both text and math mode
%
% Units
%      in text mode use 2.4\musec\ instead of 2.4 \musec\ because the space 
%        is explicitly set in the command with ``\ '' in front of mathrm
\newcommand{\seconds}{\ensuremath{\ \mathrm{s}}} % units of sec
\newcommand{\musec}  {\ensuremath{\ \mu\mathrm{s}}} % units of micro-sec
\newcommand{\msec}   {\ensuremath{\ \mathrm{ms}}} % units of millisec
\newcommand{\gcc}    {\ensuremath{\ \mathrm{g/}\mathrm{cm}^3}} % g/cm^3
\newcommand{\meters}{\ensuremath{\ \mathrm{m}}} % units of meters

% Isotopes
\newcommand{\iso}[2]{\ensuremath{^{#2}\mathrm{#1}}} % arbitrary isotope
\newcommand{\carb}{\iso{C}{12}}
\newcommand{\mg}{\iso{Mg}{24}}
\newcommand{\oxy}{\iso{O}{16}}
\newcommand{\nickel}{\iso{Ni}{56}}

% Names
\newcommand{\maestro}{{\sf MAESTRO}} % pretty output of MAESTRO
\newcommand{\mesa}{{\sf MESA}} % pretty output of MESA

% Dimensionless Numbers
\newcommand{\pr}{\ensuremath{\mathrm{Pr}}} % Prandtl #
\newcommand{\ra}{\ensuremath{\mathrm{Ra}}} % Rayleigh #

% Partial Derivs
\newcommand{\ddt}{\ensuremath{\frac{\partial}{\partial t}}} % fraction d/dt
\newcommand{\dxdt}[1]{\ensuremath{\frac{\partial #1}{\partial t}}} % frac dx/dt
\newcommand{\DDt}{\ensuremath{\frac{D}{D t}}} % fraction D/Dt
\newcommand{\DxDt}[1]{\ensuremath{\frac{D #1}{D t}}} % frac Dx/Dt

\newcommand{\ddz}{\ensuremath{\frac{\partial}{\partial z}}} % fraction d/dz
\newcommand{\dddz}{\ensuremath{\frac{\partial^2}{\partial z^2}}} % frac d2/dz2
\newcommand{\dxdz}[1]{\ensuremath{\frac{\partial #1}{\partial z}}} % frac dx/dz
\newcommand{\ddxdzz}[1]{\ensuremath{\frac{\partial^2 #1}{\partial z^2}}}

\def\ptl {\partial}
\def\del {\vec{\nabla}}

% Fractions
\newcommand{\sfrac}[2]{\mathchoice
  {\kern0em\raise.5ex\hbox{\the\scriptfont0 #1}\kern-.15em/
   \kern-.15em\lower.25ex\hbox{\the\scriptfont0 #2}}
  {\kern0em\raise.5ex\hbox{\the\scriptfont0 #1}\kern-.15em/
   \kern-.15em\lower.25ex\hbox{\the\scriptfont0 #2}}
  {\kern0em\raise.5ex\hbox{\the\scriptscriptfont0 #1}\kern-.2em/
   \kern-.15em\lower.25ex\hbox{\the\scriptscriptfont0 #2}}
  {#1\!/#2}}

\newcommand{\half}{\frac{1}{2}}
\newcommand{\myhalf}{\sfrac{1}{2}}
\newcommand{\nph}{{n + \myhalf}}
\newcommand{\nmh}{{n - \myhalf}}
\newcommand{\npo}{{n + 1}}
\newcommand{\nmo}{{n - 1}}

% Discrete Stuff
\newcommand{\dt}{{\Delta t}}
\newcommand{\dx}{{\Delta x}}
\newcommand{\dy}{{\Delta y}}
\newcommand{\dz}{{\Delta z}}
\newcommand{\dr}{{\Delta r}}

