%
% CU Boulder Spring 2015 Tayler Spruit Stuff
%
% Ryan Orvedahl Spring 2015

%\documentclass[titlepage,12pt]{article}
%\documentclass[apj]{emulateapj}
\documentclass[apj,onecolumn]{emulateapj}
%\documentclass[apj,twocolumn]{emulateapj} 
%\bibliographystyle{apj}

% Define new commands (shortcuts):
\input shortcuts

\newcommand{\uvec}{\ensuremath{\vec{u}}} % \vec{u} shortcut
\newcommand{\Bvec}{\ensuremath{\vec{B}}} % \vec{B} shortcut

% allows the \hl{} command to highlight stuff
\usepackage{color,soul}

% special colors and definitions for framing source code listings
\usepackage{listings}

\definecolor{AntiqueWhite3}{rgb}{0.804,0.753,0.69}
\definecolor{DarkerAntiqueWhite3}{rgb}{0.659,0.635,0.612}
\definecolor{orange}{rgb}{1.0,0.65,0.0}

\lstset{%
  keywordstyle=\color{blue}\ttfamily,%
  commentstyle=\itshape\color[gray]{0.5},%
  mathescape=true,%
  basicstyle=\small\ttfamily,%
  %frameround=fttt,%
  frameround=ffff,%
  %frame=shadowbox,%
  frame=single,%
  rulesepcolor=\color{DarkerAntiqueWhite3},%
  backgroundcolor=\color{AntiqueWhite3},%
  emph={load,add_slice,save}, emphstyle=\color{blue},%
  emph={[2]In}, emphstyle=[2]\color{yellow},%
  emph={[3]Out}, emphstyle=[3]\color{orange},%
  xleftmargin=1em,
  xrightmargin=1em,
  mathescape=false}

%\usepackage{pdflscape} % landscape ability
%\usepackage{scrextend} % footnote and \footref commands

\begin{document}

%========================================================================
% create title and author
%========================================================================
\title{Equation Set for Tayler-Spruit Dynamo}
\author{Ryan Orvedahl$^1$}
\affil{$^1$Department of Astrophysical \& Planetary Sciences, University 
of Colorado at Boulder, 
Boulder, CO 80309}
%\date{\today}

%%%%%%%%%%%%%%%%%%%%%%%%%%%%%%%%%%%%%%%%%%%%%%%%%%%%%%%%%%%%%%%%%%%%%%%%
% Abstract
%%%%%%%%%%%%%%%%%%%%%%%%%%%%%%%%%%%%%%%%%%%%%%%%%%%%%%%%%%%%%%%%%%%%%%%%
\begin{abstract}
We write down the equations to be used to study the Tayler-Spruit Dynamo 
problem.
\end{abstract}

\keywords{MHD, Hydrodynamics, Computational Fluid Dynamics, Instability}

\maketitle

%%%%%%%%%%%%%%%%%%%%%%%%%%%%%%%%%%%%%%%%%%%%%%%%%%%%%%%%%%%%%%%%%%%%%%%%
% Intro
%%%%%%%%%%%%%%%%%%%%%%%%%%%%%%%%%%%%%%%%%%%%%%%%%%%%%%%%%%%%%%%%%%%%%%%%
\section{Governing Equations}
\label{sec:eqns}
We model the Tayler-Spruit dynamo in a fully compressible, resistive MHD 
framework with an Ideal gas equation of state:
\begin{equation}
\DxDt{\rho} = - \rho\del\cdot\uvec
\end{equation}

\begin{equation}
P=(\gamma - 1)C_v \rho T = \rho^\gamma e^{S/C_v}
\end{equation}

\begin{equation}
\rho T \DxDt{S} = \kappa\del^2 T +
 \stackrel{\leftrightarrow}{\Pi} \mathbf{\colon} \del \uvec
\end{equation}

\begin{equation}
\dxdt{\Bvec} = \del \times \left(\uvec\times\Bvec\right) + \eta\del^2\Bvec
\end{equation}

\begin{equation}
\rho\DxDt{\uvec} = -\del\left(P + \frac{B^2}{8\pi} + \phi_{eff}\right) +
           2\uvec\times\vec{\Omega} +
           \frac{\left(\del\times\Bvec\right)\times\Bvec}{4\pi} +
           \del\cdot\stackrel{\leftrightarrow}{\Pi}
\end{equation}
Where we have used the following definitions:
\begin{equation}
\stackrel{\leftrightarrow}{\Pi} = \mu\left(\del\uvec +
       \left(\del\uvec\right)^T -
       \frac{2}{3}\stackrel{\leftrightarrow}{I}\del\cdot\uvec\right)
\end{equation}
\begin{equation}
\DDt = \ddt + \uvec\cdot\del
\end{equation}
\begin{equation}
\gamma = \frac{C_p}{C_v}
\end{equation}
\begin{equation}
\phi_{eff} = \phi + \half\lambda^2\Omega^2
\end{equation}
and $\lambda$ is the distance from the rotation axis. We have also assumed 
that the thermal, viscous and magnetic diffusivities ($\kappa$, $\mu$ and 
$\eta$) are all constant.


%%%%%%%%%%%%%%%%%%%%%%%%%%%%%%%%%%%%%%%%%%%%%%%%%%%%%%%%%%%%%%%%%%%%%%%%
% Thermo
%%%%%%%%%%%%%%%%%%%%%%%%%%%%%%%%%%%%%%%%%%%%%%%%%%%%%%%%%%%%%%%%%%%%%%%%
\section{Thermodynamics}
\label{sec:thermo}
We start with the Ideal gas equation of state and solve it for the entropy:
\begin{equation}
\frac{S}{C_v} = \log P - \gamma \log \rho \Rightarrow
\frac{S}{\gamma C_v} =
         \frac{1}{\gamma}\log\left((\gamma - 1)C_v\rho T\right) - \log\rho
\end{equation}
\begin{equation}
\frac{S}{C_p} = \frac{1}{\gamma}\log\left((\gamma - 1)C_v\right) +
              \left(\frac{1}{\gamma}-1\right)\log\rho +
              \frac{1}{\gamma}\log T
\end{equation}
Assuming both $C_v$ and $\gamma$ are constant,
\begin{equation}
\frac{1}{C_v}\DxDt{S} = (1-\gamma)\DxDt{\log\rho} + \DxDt{\log T} =
 (\gamma -1)\del\cdot\uvec + \DxDt{\log T}
\end{equation}
Where we have used the continuity equation for the last equality.


%%%%%%%%%%%%%%%%%%%%%%%%%%%%%%%%%%%%%%%%%%%%%%%%%%%%%%%%%%%%%%%%%%%%%%%%
% Linearize
%%%%%%%%%%%%%%%%%%%%%%%%%%%%%%%%%%%%%%%%%%%%%%%%%%%%%%%%%%%%%%%%%%%%%%%%
\section{Linearized Equations}
\label{sec:linear}
To linearize the equations, we decompose the fields into background and 
fluctuating parts: $x(\vec{r},t) = x_0(\vec{r}) + x_1(\vec{r},t)$. We further 
assume that the fluctuating parts are small compared to the background:
$|x_1| \ll |x_0|$. This allows us to drop any terms that are quadratic in 
the fluctuating quantities, i.e. terms that would appear as $x_1 y_1$. We also 
impose that the background quantites satisfy the equations. This allows us 
to subtract the background state after plugging in the decomposed fields, 
leaving equations for the evolution of the fluctuations.

First we linearize the equation of state using the assumption that $x_1/x_0$ 
is a small quantity and the approximation $\log(1+\epsilon) \approx \epsilon$.
\begin{equation}
P = (\gamma -1)C_v\rho T
\end{equation}
\begin{equation}
P_1+P_0 = (\gamma -1)C_v\left(\rho_0+\rho_1\right)\left(T_0+T_1\right)
\end{equation}
\begin{equation}
P_0\left(1+\frac{P_1}{P_0}\right) = (\gamma -1)C_v\rho_0 T_0
          \left(1+\frac{\rho_1}{\rho_0}\right)
          \left(1+\frac{T_1}{T_0}\right)
\end{equation}
\begin{equation}
\left(1+\frac{P_1}{P_0}\right) = \left(1+\frac{\rho_1}{\rho_0}\right)
                                 \left(1+\frac{T_1}{T_0}\right)
\end{equation}
\begin{equation}
\log\left(1+\frac{P_1}{P_0}\right) = \log\left(1+\frac{\rho_1}{\rho_0}\right)
                                 + \log\left(1+\frac{T_1}{T_0}\right)
\end{equation}
\begin{equation}
\frac{P_1}{P_0} = \frac{\rho_1}{\rho_0} + \frac{T_1}{T_0}
\end{equation}

The final set of linearized equations is:
\begin{equation}
\dxdt{\rho_1} = -\rho_0\del\cdot\uvec_1 - \uvec_1\cdot\del\rho_0
\end{equation}
\begin{equation}
\dxdt{\Bvec_1} = \del\times\left(\uvec_1\times\Bvec_0\right)+\eta\del^2\Bvec_1
\end{equation}
\begin{equation}
\rho_0\dxdt{T_1} = -\rho_0\uvec_1\cdot\del T_0 -
                     \rho_0 T_0 (\gamma -1)\del\cdot\uvec_1 +
                     \frac{\kappa}{C_v}\del^2T_1
\end{equation}
\begin{equation}
\rho_0\dxdt{\uvec_1} = -\del\left(P_1 +\frac{B_0B_1}{4\pi} + \phi_{eff}\right)
                       +2\uvec_1\times\vec{\Omega} +
          \frac{\left(\del\times\Bvec_0\right)\times \Bvec_1}{4\pi} +
          \frac{\left(\del\times\Bvec_1\right)\times \Bvec_0}{4\pi} +
          \mu\del^2\uvec_1 + \frac{\mu}{3}\del\left(\del\cdot\uvec_1\right)
\end{equation}
\begin{equation}
\frac{P_1}{P_0} = \frac{\rho_1}{\rho_0} + \frac{T_1}{T_0}
\end{equation}


%%%%%%%%%%%%%%%%%%%%%%%%%%%%%%%%%%%%%%%%%%%%%%%%%%%%%%%%%%%%%%%%%%%%%%%%
% Conclusions
%%%%%%%%%%%%%%%%%%%%%%%%%%%%%%%%%%%%%%%%%%%%%%%%%%%%%%%%%%%%%%%%%%%%%%%%
%\section{Conclusions and Discussion}
%\label{sec:conclusions}
%We presented results of numerically integrating the equations of motion 

%%%%%%%%%%%%%%%%%%%%%%%%%%%%%%%%%%%%%%%%%%%%%%%%%%%%%%%%%%%%%%%%%%%%%%%%
% Acknowledgements
%%%%%%%%%%%%%%%%%%%%%%%%%%%%%%%%%%%%%%%%%%%%%%%%%%%%%%%%%%%%%%%%%%%%%%%%
%\section*{Acknowledgements}
%\label{sec:acks}
%I would like to thank my computer for not crashing during the data collection 
%phase of this project.

%%%%%%%%%%%%%%%%%%%%%%%%%%%%%%%%%%%%%%%%%%%%%%%%%%%%%%%%%%%%%%%%%%%%%%%%
% References
%%%%%%%%%%%%%%%%%%%%%%%%%%%%%%%%%%%%%%%%%%%%%%%%%%%%%%%%%%%%%%%%%%%%%%%%
% include references
%\nocite{*}
%\bibliography{Bibliography}


%%%%%%%%%%%%%%%%%%%%%%%%%%%%%%%%%%%%%%%%%%%%%%%%%%%%%%%%%%%%%%%%%%%%%%%%
% Figures
%%%%%%%%%%%%%%%%%%%%%%%%%%%%%%%%%%%%%%%%%%%%%%%%%%%%%%%%%%%%%%%%%%%%%%%%

%======================================================================
% the figure* environment allows the figure to take up two columns
%\begin{figure*}[h]
%\begin{center}
%\centerline{\includegraphics[scale=0.4,angle=0]{./Plots/.pdf}}
%\caption{}
%\label{fig:}
%\end{center}
%\end{figure*}

%%%%%%%%%%%%%%%%%%%%%%%%%%%%%%%%%%%%%%%%%%%%%%%%%%%%%%%%%%%%%%%%%%%%%%%%
% Data
%%%%%%%%%%%%%%%%%%%%%%%%%%%%%%%%%%%%%%%%%%%%%%%%%%%%%%%%%%%%%%%%%%%%%%%%

%======================================================================
%\begin{table}[h]
%\begin{center}
%\begin{tabular}{|c|c|c|c|}
%\hline
%& Simple-3 (\maestro) & Regular-9 (\mesa)& Extended-33 (\mesa) \\ \hline
%1 & C12 & H1 & neutron \\ \hline
%2 & O16 & He4 & H1 \\ \hline
%3 & Mg24 & C12 & He3 \\ \hline
%4 & - & O16 & He4\\ \hline
%5 & - & Ne20 & Be9\\ \hline
%6 & - & Na23 & C12\\ \hline
%\end{tabular}
%\caption{List of isotopes used in each network. The Regular-9 and Extended-33 
%networks used the \mesa\ code while the Simple-3 network used the \maestro\ 
%code.}
%\label{tab:isos}
%\end{center}
%\end{table}

\end{document}


